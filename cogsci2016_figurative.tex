% 
% Annual Cognitive Science Conference
% Sample LaTeX Two-Page Summary -- Proceedings Format
% 

% Original : Ashwin Ram (ashwin@cc.gatech.edu)       04/01/1994
% Modified : Johanna Moore (jmoore@cs.pitt.edu)      03/17/1995
% Modified : David Noelle (noelle@ucsd.edu)          03/15/1996
% Modified : Pat Langley (langley@cs.stanford.edu)   01/26/1997
% Latex2e corrections by Ramin Charles Nakisa        01/28/1997 
% Modified : Tina Eliassi-Rad (eliassi@cs.wisc.edu)  01/31/1998
% Modified : Trisha Yannuzzi (trisha@ircs.upenn.edu) 12/28/1999 (in process)
% Modified : Mary Ellen Foster (M.E.Foster@ed.ac.uk) 12/11/2000
% Modified : Ken Forbus                              01/23/2004
% Modified : Eli M. Silk (esilk@pitt.edu)            05/24/2005
% Modified : Niels Taatgen (taatgen@cmu.edu)         10/24/2006
% Modified : David Noelle (dnoelle@ucmerced.edu)     11/19/2014

%% Change "letterpaper" in the following line to "a4paper" if you must.

\documentclass[10pt,letterpaper]{article}

\usepackage{cogsci}
\usepackage{pslatex}
\usepackage{apacite}
\usepackage{todonotes}
\usepackage[usenames, dvipsnames]{color}


\title{Empirical and Computational Approaches to Metaphor and Figurative Meaning}
 
\author{{\large \bf Justine T. Kao (justinek@stanford.edu) and Noah D. Goodman (ngoodman@stanford.edu)} \\ 
  Department of Psychology, 
  Stanford University, Stanford CA, USA 
  \AND{
\large \bf Francisco Maravilla (fmaravil@gmail.com) and Dedre Gentner (gentner@northwestern.edu)} \\
  Department of Psychology, Northwestern University,
  Evanston IL, USA
  \AND{\large \bf Gerard Steen (g.j.steen@uva.nl)} \\
  Department of Dutch, University of Amsterdam,
Amsterdam, Netherlands
\AND{\large \bf Tony Veale (tony.veale@ucd.ie)} \\
  Department of Computer Science, University College Dublin,
  Dublin, Ireland
}


\begin{document}

\maketitle

\begin{quote}
\small
\textbf{Keywords:} 
Figurative language; Metaphor; Computational modeling; Computational linguistics; Psycholinguistics
\end{quote}

\section{Motivation}
One of the hallmarks of human intelligence is the ability to go beyond literal meanings of utterances to infer speakers' intended meanings, a feat that remains elusive to the most advanced artificial systems. Figurative language such as metaphor, in particular, provides a striking case where complex meanings arise that cannot be derived from the literal semantics alone. For example, \emph{My lawyer is a shark} is a literally false sentence, yet communicates relevant features of the lawyer in question (e.g. \emph{ruthless}, but not \emph{swims}). 
%Examining how people derive and produce figurative meanings has been an active area of research in psychology, linguistics, philosophy, and, more recently, computer science. 

%These fields have generated a great deal of work concerning the relationship between metaphor and thought \cite{gibbs2008metaphor, lakoff1993contemporary}, metaphor and analogical reasoning \cite{gentner2001metaphor}, figurative language and communication, and the relationship between figurative meaning and creativity.

Metaphor and other types of figurative language raise a range of questions core to the study of language and cognition;
%, such as the relationship between metaphor and thought, analogical reasoning, communication, and creativity. 
as a result, how people derive and produce figurative meanings has been approached from various angles and with a diverse set of methods. Some psychologists focus on the cognitive mechanisms that underly interpretations of specific types of figurative use, such as how people align shared properties and analogous relations across domains in order to understand metaphor \cite{gentner1983structure}. 
%\textcolor{red}{Others aim to identify figurative uses in corpora in order to extract patterns that shed light on how figurative meanings are generated and interpreted in naturalistic contexts \cite{steen2010method}}. 
Other researchers apply theories of communication to explain how people arrive at contextually appropriate interpretations of non-literal utterances \cite{steen2015developing, kao2014nonliteral}. Finally, natural language processing researchers seek to identify features and principles that can help artificial agents process and produce figurative language \cite{veale2007comprehending}.

In this symposium, we will discuss the methods that our speakers have employed to examine how people interpret and produce figurative meaning, as well as ways to combine complementary approaches. By bringing together experts with different theoretical perspectives and from a range of disciplines, we aim to discuss outstanding questions that may require a synthesis of tools to resolve. We will open the symposium with a brief overview of the landscape of metaphor and figurative meaning, provided by \textbf{N. Goodman}. We will then present four 20-minute talks, starting with \textbf{F. Maravilla} presenting joint work with \textbf{D. Gentner} on \textcolor{red}{[insert summary]}.
\textbf{G. Steen} will then introduce the Deliberate Metaphor Theory and highlight the importance of communicative intent. Continuing the thread, \textbf{J. Kao} will present joint work with \textbf{Goodman} on applying Bayesian models of communication to interpret figurative uses. Finally, \textbf{T. Veale} will present his work on a computational system that employs cognitive principles to automatically generate figurative language.
The symposium will end with a 20-30 minute discussion with the speakers and audience (facilitated by \textbf{Goodman} and \textbf{Kao}).	


\textcolor{red}{\section{Title}}
\large \textbf{Francisco Maravilla \& Dedre Gentner}\\
\textcolor{red}{Abstract}

\section{Introducing Deliberate Metaphor Theory}
\large \textbf{Gerard Steen}\\
%\textcolor{red}{Abstract}
99\% of all metaphorically used words have meanings that are so conventionalized that they can be found straight in a user's dictionary \cite{steen2010method}. This corpus-linguistic finding raises fundamental questions about how people produce and interpret metaphor in discourse. For instance, if conventionalized metaphorical meanings are so readily available, can a processing model that includes lexical disambiguation between metaphorical and non-metaphorical senses account for most metaphor use? This would go against Conceptual Metaphor Theory \cite{lakoff2008metaphors} and support part of \citeA{sperber2008deflationary}'s deflationary account of metaphor and would agree with \citeA{giora2008metaphor}'s position that (most) metaphor is nothing special. However, not all metaphor is conventional, and even conventional metaphor can require processing via on-line cross-domain comparison \cite{steen2008paradox, steen2015developing}. This is when metaphor is used deliberately as a metaphor, the referential meaning of the utterance in context involving an intentional comparison between two concepts, categories, spaces or domains. Examples include all novel metaphors (where processing can only go via the meaning of the source domain to arrive at an interpretation of the utterance as a whole) and all explicit comparisons (such as similes and analogies where the target is explicitly compared to some alien source). These deliberate metaphors point to a third dimension of metaphor, next to language and thought, i.e., communication, and can explain some of the current findings and theories, including \citeA{bowdle2005career}'s  Career of Metaphor Theory, in alternative ways. The paper will present a brief account of Deliberate Metaphor Theory.

\section{Modeling figurative communication}
\large \textbf{Justine T. Kao \& Noah D. Goodman}\\
Figurative language often requires listeners to access background knowledge and discourse context in order to arrive at appropriate interpretations. 
%For example, the metaphor \emph{John is a rock} is taken as a compliment when the topic of conversation is John's stability, but as criticism when the topic is his intellectual ability. 
How do listeners incorporate multiple sources of information to derive true information from literally false utterances? We describe an extended version of the Rational Speech Act (RSA) model, a Bayesian computational model that formalizes core principles of communication. Our model highlights the idea that speakers are assumed to be informative with respect to specific communicative goals, which may be to convey subjective attitudes instead of objective information about the world. By reasoning about the range of communicative goals a speaker may have, listeners can go beyond the literal meanings of utterances to arrive at figurative interpretations. Through a series of experiments, we show that our model predicts people's interpretations of hyperbole, irony, and metaphor with high accuracy. We show that despite apparent differences among these subtypes of figurative language, the same computational framework flexibly produces fine-grained interpretations for diverse figurative uses. We use this as evidence suggesting that the rich and often affectively-laden meanings expressed by figurative language can be explained by basic principles of communication.


\section{Metaphor and the human creative potential}
\large \textbf{Tony Veale}

Metaphors in the wild rarely occur in isolation. Rather, metaphors in many texts are supported by, or give support to, other metaphors, similes and blends.  These various figurative devices present different affordances to an author, and serve complementary roles in the communication of complex ideas. A tentative simile or analogy may lay the groundwork for bolder metaphors to come, which can in turn set the stage for an immersive conceptual blend. Clearly, these are not disjoint phenomena arising from distinct mechanisms, but the manifestations of different settings --- such as degree of integration --- of a common creative mechanism. Computational modeling offers researchers a generative approach to demonstrating this claim, by supporting the construction of generative systems that can produce their own similes, metaphors, analogies and blends from a small set of core principles. In this talk I explore the workings of such a computational system, called Metaphor Magnet, that is realized in a number of public forms, from a Twitterbot called @MetaphorMagnet to a Web service that provides figurative competence to 3rd-party software systems. In particular, I show how a separation of concerns --- chiefly, of conceptual content and linguistic framing --- allows the system to achieve a wide variety of human-like figurative outputs from a small number of cognitive principles and AI mechanisms.


\bibliographystyle{apacite}

\setlength{\bibleftmargin}{.125in}
\setlength{\bibindent}{-\bibleftmargin}

\bibliography{symposium}


\end{document}
